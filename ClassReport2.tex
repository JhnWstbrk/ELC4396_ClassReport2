% Digital Logic Report Template
% Created: 2020-01-10, John Miller

%==========================================================
%=========== Document Setup  ==============================

% Formatting defined by class file
\documentclass[11pt]{article}

% ---- Document formatting ----
\usepackage[margin=1in]{geometry}	% Narrower margins
\usepackage{booktabs}				% Nice formatting of tables
\usepackage{graphicx}				% Ability to include graphics

%\setlength\parindent{0pt}	% Do not indent first line of paragraphs 
\usepackage[parfill]{parskip}		% Line space b/w paragraphs
%	parfill option prevents last line of pgrph from being fully justified

% Parskip package adds too much space around titles, fix with this
\RequirePackage{titlesec}
\titlespacing\section{0pt}{8pt plus 4pt minus 2pt}{3pt plus 2pt minus 2pt}
\titlespacing\subsection{0pt}{4pt plus 4pt minus 2pt}{-2pt plus 2pt minus 2pt}
\titlespacing\subsubsection{0pt}{2pt plus 4pt minus 2pt}{-6pt plus 2pt minus 2pt}

% ---- Hyperlinks ----
\usepackage[colorlinks=true,urlcolor=blue]{hyperref}	% For URL's. Automatically links internal references.

% ---- Code listings ----
\usepackage{listings} 					% Nice code layout and inclusion
\usepackage[usenames,dvipsnames]{xcolor}	% Colors (needs to be defined before using colors)

% Define custom colors for listings
\definecolor{listinggray}{gray}{0.98}		% Listings background color
\definecolor{rulegray}{gray}{0.7}			% Listings rule/frame color

% Style for Verilog
\lstdefinestyle{Verilog}{
	language=Verilog,					% Verilog
	backgroundcolor=\color{listinggray},	% light gray background
	rulecolor=\color{blue}, 			% blue frame lines
	frame=tb,							% lines above & below
	linewidth=\columnwidth, 			% set line width
	basicstyle=\small\ttfamily,	% basic font style that is used for the code	
	breaklines=true, 					% allow breaking across columns/pages
	tabsize=3,							% set tab size
	commentstyle=\color{gray},	% comments in italic 
	stringstyle=\upshape,				% strings are printed in normal font
	showspaces=false,					% don't underscore spaces
}

% How to use: \Verilog[listing_options]{file}
\newcommand{\Verilog}[2][]{%
	\lstinputlisting[style=Verilog,#1]{#2}
}




%======================================================
%=========== Body  ====================================
\begin{document}

\title{System on Chip: Class Report 2}
\author{Noel Sengel and John Westbrook}

\maketitle


\section*{Summary}

In this class report, our goal was to create a reaction game using an LED on our NEXYS board. Once the LED lit up, a timer started to see how long it took the person playing to press a button and react. To accomplish this, we took code from a previous class which converted binary to Binary Coded Decimal and displayed it on the 7 segment display. Then, we added our own code and editted the modules to display the millisecond timer while it was running. Our main development occured in one module, "reaction\_timer".


\section*{Results}

Here is a test we did for reaction\_timer module which ensures its functionality
\begin{figure}[ht]\centering
	\includegraphics[width=\textwidth]{"WaveformClassReport2"}
	\caption{reaction\_timer Simulation}
	\label{fig1}
\end{figure}


\section*{Code}

Here is the GitHub repo with our modules: 
\url{https://github.com/JhnWstbrk/ELC4396_ClassReport2}

In Listing 1, you can see a section of our main module, "reaction\_timer", which is the main logic that implements the game part.
\begin{lstlisting}[style=Verilog,
caption=Main Logic of reaction\_timer,
label=code:ex 
]
...
	always_comb begin
	        if(reaction_state == START) begin
	              led = 1'b1;
	              led_on = 1'b1;
	               timer_start = 1'b1;
	        end    
	        if(reaction_state == STOP && timer_start == 1'b1) begin
	             //display last time on the screen
	            timer_start = 1'b0;
	            led = 1'b0;
	        end
	        if(timer == 1000) begin
		        timer_start = 1'b0;
		        led = 1'b0;
		    end
		    
	        if(reaction_state == STOP && led_on != 1'b1) begin
	            
	        end
	        if(reaction_state == CLEAR || rst == 1'b1) begin
	            led = 1'b0;
	            
	        end
	    end
...
\end{lstlisting}


\end{document}
